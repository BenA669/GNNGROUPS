\documentclass[preview]{standalone}
\usepackage[english]{babel}
\usepackage{amsmath}
\usepackage{amssymb}
\begin{document}
\begin{align*}
tensor([ 1.8962, -0.4820], device='cuda:0')
\end{align*}
\end{document}